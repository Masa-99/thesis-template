\thispagestyle{empty}
\section*{Kurzdarstellung}
\label{sec:kurzdarstellung}
In der vorliegenden Bachelorarbeit wird untersucht, ob und welche Vor- und Nachteile das Entwickeln einer Progressive Web App (PWA) -- eine Webanwendung, die Funktionalitäten wie Offlinebetrieb und Installation unterstützt -- im Gegensatz zu einer Native App bieten. 
Ferner soll damit die Frage beantwortet werden, ob sie diese ersetzen kann, um die Chancen von PWAs zu beurteilen.
Um eine Grundlage für den Vergleich zu schaffen, findet hierbei eine plattformunabhängige Entwicklung der Native App statt.\\
Für den Vergleich wurde auf Basis eines Kriterienkatalogs, bestehend aus Funktionalität, Kompatibilität mit verschiedenen Betriebssystemen und Entwicklungsaufwand, jeweils eine App entwickelt.
Diese stellt die aktuellen COVID-19 Fallzahlen dar und unterstützt neben der Installation und dem Offlinebetrieb die Funktionen Standortzugriff, Kontaktzugriff und Benachrichtigungen.\\
Die implementierten Funktionalitäten verdeutlichen, dass PWAs durch moderne Webschnittstellen an Potential gewinnen.
Vorteile sind dabei die Unabhängigkeit von der Installation der Anwendung und der geringere Entwicklungsaufwand.\\
Jedoch ist eine entscheidende Schwachstelle von PWAs die fehlende Unterstützung einiger Funktionalitäten auf iOS Geräten.\\
Zusammenfassend zeigt das Ergebnis, dass PWAs nur bis zu einem gewissen Grad in der Lage sind, nativen Anwendungen zu ersetzen.
Deshalb muss die Entscheidung für die Entwicklung einer PWA zudem abhängig von den Anforderungen des Anwendungsfalls betrachtet werden.