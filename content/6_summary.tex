\chapter{Fazit und Ausblick}\label{ch:summary}
\section{Fazit}
In dieser Arbeit wurden anhand der Kriterien Funktionalität, Kompatibilität mit verschiedenen Betriebssystemen und Entwicklungsaufwand \acp{pwa} mit Native Apps verglichen.
Letzteres wurde mit React Native, einem plattformunabhängigen Framework zur Entwicklung von nativen Anwendungen, entwickelt, um eine faire Grundlage für den Vergleich zu schaffen.
Für den Vergleich wurde mit diesen Technologien jeweils eine App entwickelt, welche die COVID-19-Fallzahlen darstellt und filtern kann.
Dabei wurden außerdem die Funktionalitäten Installation, Offlinebetrieb, Standortzugriff, Kontaktzugriff und Benachrichtigungen implementiert.
Anhand der Programmierung dieser App wurde analysiert, welche Vor- und Nachteile das Entwickeln mit diesen Technologien hat.
Auf diesen Beobachtungen aufbauend soll geschlussfolgert werden, ob die \ac{pwa} die Native App ersetzen kann.

Dabei hat sich herausgestellt, dass \acp{pwa} mittlerweile viele Funktionalitäten von nativen Anwendungen umsetzen können und deshalb gerade für simple Anwendungen eine Alternative bieten.
Gerade bei der Installation weist die \ac{pwa} im Vergleich zur Native App Stärken auf.
Außerdem ist der Entwicklungsaufwand der \ac{pwa} geringer als der einer nativen Anwendung.

Dennoch ist ein entscheidender Faktor gegen die Etablierung von \acp{pwa} die fehlende Unterstützung durch iOS, da deshalb etwa 26 Prozent des weltweiten Marktanteils von mobilen Betriebssystemen weniger Funktionalitäten zur Verfügung stehen \cite{ODea.2021}.
Dies betrifft wie in der vorliegenden Arbeit dargestellt die Push \ac{api} und Notification \ac{api}, wodurch die das Empfangen von Benachrichtigungen mit iOS Geräten nicht möglich ist.\\
Doch in Anbetracht der Vielzahl von Webschnittstellen, die sich seit dem Aufschwung von \acp{pwa} 2015 entwickelt haben, wird deutlich, dass die Zukunft von mobilen Anwendungen vielfältig ist.

Zusammenfassend legen die Ergebnisse nahe, dass eine Entscheidung für eine \ac{pwa} oder eine Native App abhängig vom Anwendungsfall der App getroffen werden sollte.

\section{Ausblick}
Für anknüpfende Arbeiten könnten die Analyse von \acp{pwa} und Native Apps intensiviert werden, indem umfassender auf die verschiedenen Funktionalitäten eingegangen wird, die mit diesen beiden Technologien umsetzbar sind.
Beispiele hierfür sind die Einbindung der Background Sync \ac{api}, Payment Request \ac{api} oder Web Share Target \ac{api}.\\
Zusätzlich dazu kann aufgrund der stetigen Verbesserung der letzten Jahre angenommen werden, dass eine Erweiterung des Angebots von Webschnittstellen in Zukunft stattfindet.
Deshalb ist eine erneute Betrachtung der Thematik für die Beurteilung, ob \acp{pwa} Native Apps ersetzen können, unabdingbar.

Interessant wäre außerdem ein Performance Vergleich der zwei Technologien.
Da die für diese Arbeit implementierte Anwendung lediglich aus einer Seite mit Daten einer externen Schnittstelle besteht, hat sich der Vergleich nicht angeboten.
Anders sieht das jedoch bei größeren Anwendungen aus, welche eine Vielzahl von Bildern, Videos und Einträgen nutzen und verwalten wie Twitter oder Pinterest.
Auch die Ansprache der Schnittstellen eines mobilen Endgeräts sind unter dem Aspekt der Performance zu betrachten, denn dies geschieht performanter mit der betriebssystemspezifischen Programmiersprache.

Ein weiterer wichtiger Faktor, der in zukünftigen Arbeiten betrachtet werden sollte, ist die Sicherheit von Progressive Web Apps.
Generell besteht eine grundlegende Sicherheit durch den Zugriff mit \ac{https}, jedoch sollte hier auch die umfassend betrachtet werden, welche Maßnahmen zur Verbesserung der Sicherheit von Webanwendungen getroffen werden können.
Durch die Position des Service Workers als Proxy ist dieser besonders anfällig für bösartige Angriffe \cite{Lee.2018}.
Der Sicherheitsaspekt ist vor allem bedeutsam für mobilen Anwendungen, die sensible Daten verarbeiten.