\chapter{Ergebnisse und Diskussion}\label{ch:results}
Die Evaluierung der beiden Entwicklungsansätzen erfolgt über den bereits erläuterten Kriterienkatalog. Die Ergebnisse der Implementierung sind durch das Testen beider Anwendungen auf einem Apple IPhone 12 und einem Samsung S10 entstanden.

•	Wars möglich alles umzusetzen?
•	Gab es irgendwo irgendwelche Beschränkungen?
•	Nochmal Kriterienkatalog dafür, ob alle Funktionaltäten passen
•	Tabelle für Vergleich insgesamt	?
Keine Bewertungen, wenn bei Funktionalität bei beiden alles geht
•	Unterschiede zwischen Implementierungen
•	Vielleicht noch was reinbringen, was bei pwas nicht funktioniert
•	Was funktioniert und was nicht bei PWAs und RN? 


\section{Funktionalität}
In beidem gut umsetzbar. Bei der React Native App konnten die geforderten Funktionalitäten sowohl mit Expo als auch mit reinem RN einfach implementiert werden. Der plattformspezifische Code bedarf jedoch wiederum genauer Auseinandersetzung mit der jeweiligen Programmierung der Funktionalitäten.

Installierbarkeit
Offline-Betrieb
Standortzugriff
Benachrichtigungen



\section{Plattformunabhängigkeit}
Hier das Testing anbringen?
PWA
-	iOS: 1. Öffnen der PWA im Safari Browser. 2. Auf das Teilen Icon klicken 3. „Zum Home-Bildschirm“. getestet auf einem Iphone 12, iOS 14.4. Iphone mit Chrome funktioniert es nicht.
-	Android: 1. Öffnen der PWA im Chrome Browser 2. Auf 3 Punkte (Einstellungen) klicken 3. „App installieren“ 4. Im Prompt auf Installieren klicken“
1. Öffnen der PWA im „eingebauten“ Browser 2. Prompt kommt || „Hinzufügen zu…“ Aus-wahlmöglichkeit zwischen Lesezeichen, Browser-Startseite und Telefon Startbildschirm || In-formationen über Website 
-	Getestet auf einem Huawei P10 Lite, Android 10.	
Nach dem Installieren ist die PWA durch den Home Screen erreichbar. Jedoch gibt es keine au-tomatischen Aktualisierungen der Seite, falls eine neue Version veröffentlicht wird (?).
RN
-	Eigentlich allgemein Plattformunabhängig
-	Durch ejecten von expo bleibt es immer noch plattformunabhängig, aber es lassen sich auch jeweils (im gleichen Projekt!) android oder ios spezifische Implementierungen vornehmen

Der besondere Vorteile bei PWA ist, dass sie im Gegensatz zu nativen Anwendungen, selbst wenn diese drauf ausgelegt sind plattformunabhängig zu sein, dennoch unbeschränkteren Zugriff anbieten.


\section{Entwicklungsaufwand}
Um die beiden Ansätze besser vergleichen zu können, wurde einerseits die Dauer der Einarbeitung und Recherchen gemessen und andererseits die reine Implementierungsdauer. Die Anwendungen wurden mit einem soliden Grundwissen in HTML, CSS und JavaScript entwickelt. Spezifische Kenntnisse über die Frameworks waren nicht gegeben. In der folgenden Grafik wird die Auflistung der Zeit dargestellt:
PWA:
Funktionalität	Dauer - Recherche	Dauer - Implementierung
Grundlegendes Setup	1 Stunde	25min
Installierbarkeit	2 Stunden	30min
Standortabfrage	1 Stunde	15min
Benachrichtigung	5 Stunden	120min
Gesamt	9 Stunden	190min 

RN: 
Funktionalität	Dauer - Recherche	Dauer - Implementierung
Grundlegendes Setup	4 Stunde	60min
Installierbarkeit		
Standortabfrage		
Benachrichtigung
Hierbei wird deutlich, dass für die Einfindung und Implementierung der React Native App im Gegensatz zu React PWA deutlich mehr Zeit beansprucht wurde, obwohl Teile des Codes wiederverwendet werden konnten. Besonders ausschlaggebend war dabei die Einarbeitung in die Technologie, vor allem als es um das Programmieren von plattformspezifischem Code ging. Bei Betracht der im Kriterienkatalog dargelegten Gehälter eines Entwicklers, kann man ableiten, dass es für die Anforderungen an diese Anwendung kostengünstiger ist eine PWA zu entwickeln.


\section{Vorteile von PWAs gegenüber Native Apps und Grenzen, Diskussion}
\url{https://developer.mozilla.org/en-US/docs/Web/Progressive_web_apps/Introduction}
-	Schnelles installieren, weniger Speicherverbrauch
-	Aktualisierungen müssen nicht extra installiert werden aka die ganze Applikation muss nicht neu heruntergeladen werden
-	Deep Linking auf bestimmte Seiten (ist das auch möglich in normalen Apps?)
-	Bestehende Webseiten / Webapplikationen können mit wenigen Schritten in PWAs umgewan-delt werden (Installierbarkeit und Offline Modus)
Der Vorteil der Installation einer PWA gegenüber die einer Native App ist, dass sie nutzerfreundlicher ist. Bei Native App wird der Nutzer muss der Nutzer überzeugt werden, dass ihm die App den Speicherplatz wert ist, auch wenn er die Anwendung eventuell nicht oft nutzt. Bei PWAs ist der Vorteil, dass sie ein-fach und schnell bei vermehrter Nutzung der App installiert werden können und weniger Bandweite verbrauchen [6].
Zusammenfassend lässt sich sagen, das PWAs immer wichtiger werdende Konkurrenten für Native Apps sind, denn mittlerweile fehlen nur noch ein paar Funktionalitäten, um in Zukunft komplett auf Native Apps verzichten zu können. Dennoch sind die fehlenden Ausstattungsmerkmale ausschlaggebend, denn Apps ohne beispielsweise Zugriff auf Kontakte oder die Funktion des Geofencings, sind in vielen Fällen nicht konkurrenzfähig. Durch die Schnelligkeit und einfach Umsetzung einer Webb Anwendung mit er-weiterter Grundfunktionalität ist eine Progressive Web App eine solide Wahl, um schnell ans Ziel zu kommen. Dadurch dass React PWAs und React Native Apps in ihrer Grundlage auf dem gleichen Prinzip beruhen, nämlich React, ist es durch React Native möglich, eine bereits bestehende React PWA durch Übertragung in React Native zu einer Native App umzuprogrammieren, die alle Funktionalitäten einer Native App. Dies ist besonders vorteilhaft, da die Entwicklung nicht von Anfang an gemacht werden muss und sich somit viel Zeit sparen lässt. Wenn man lediglich bei dem „bare“ Workflow von React Native bleibt, hat man schnell eine Anwendung, die meist ohne weitere Konfiguration sowohl auf Android und iOS läuft. Soll die App nun noch plattformspezifische Elemente beinhalten, kann sogar durch das Verwerfen der Expo CLI komplett nativer Code entwickelt werden. Diese schrittweise Erweiterung der Möglichkeiten bietet dem Team eine enorme Flexibilität und optimiert den Entwicklungsaufwand.

\subsection{Grenzen von Progressive Web Apps}
Obwohl es mittlerweile viele Funktionalitäten gibt, die PWAs umsetzen können, gibt es immer noch einige Einschränkungen. Unter der Seite \url{https://whatwebcando.today/} ist übersichtlich dargestellt, um welche konkreten Features es sich dabei handelt. Im Bereich „Native Behaviors“ fehlt nach Ansicht des Autors nur noch die „User Idle Detection“. Gravierend ist jedoch, dass bislang nicht auf die Kontakte oder SMS / MMS zugegriffen werden kann. Hierfür ist bereits ein inoffizieller Entwurfsvorschlag bei der W3C eingegangen \cite{Beverloo.2021}. Weitere fehlende Funktionalitäten sind das Geofencing, Shape Detection bei Kameraaufnahmen und NFC Übertragung.\\
Wichtig ist dabei, dass diese Features nicht immer in allen Browsern gleichermaßen verfügbar sind. Beispielsweise ist die automatische Aufforderung zur Installation der PWA, bezeichnet als "BeforeInstallPromptEvent API", bisher nur Android Geräten und den meisten Chromium-basierten Browsern vorenthalten \cite{}. Weitere Beispiele mit unterschiedlichen Browser Support sind Background Sync, Bluetooth und Push Benachrichtigungen.
Vorteil ist hier, dass das Versprechen, dass bei Crossplatform Anwendungen genannt wird "Code once, run everywhere" gehalten wird. Einbüßungen sind dabei aber die teilweise fehlenden Funktionalitäten. Aktuell sind aber Entwürfe für x, x und x beim W3C eingereicht, weswegen in Zukunft mehr Funktionalitäten auch im Web zur Verfügung stehen und somit wirklich nur in JavaScript entwickelt werden muss.

\subsection{Grenzen von Cross-platform Apps}
eigentlich nicht immer so einfach der write once run everywhere ansatz, da oft trotzdem code extra geschrieben werden muss für bestimmte betriebssysteme
If you hire a JavaScript developer to work on your React Native project, expect that they will have to write native code to bridge any gaps in functionality.
