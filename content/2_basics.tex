\chapter{Grundlagen}\label{ch:basics}
In diesem Kapitel werden die für diese Arbeit notwendigen Grundlagen aufgezeigt. Diese sollen für ein einheitliches Verständnis des Themas sorgen und unterschiedliche Vorstellungen von Fachbegriffen angleichen.\\
Generell lassen sich mobile Anwendungen in drei Kategorien einteilen: Native, Hybrid und Web Applikationen. Im Folgenden wird einerseits genauer auf Native und Hybride Applikationen und anderseits auf eine spezielle Form von Web Apps, genannt Progressive Web Apps, eingegangen.

\section{Native Applikationen}
Als native Applikationen bezeichnet man Anwendungen, die plattformspezifisch -- ergo speziell für ein Betriebssystem -- implementiert werden. Das wird dadurch ermöglicht, dass sie mit dem Software Development Kit (SDK) des Plattformherstellers entwickelt werden und somit kompletten Zugriff auf jegliche Funktionalitäten der Geräte besitzen. Außerdem entsteht durch Verwendung der plattformspezifischen UI-Komponenten\footnote{User Interface Komponenten, z. Dt. Benutzeroberflächenkomponenten}, sogenannten Views, eine einheitliche Benutzerschnittstelle, die sich in allen Anwendungen des Betriebssystems widerspiegelt. Um solche Apps mit ihrem SDKs zu entwickeln, wird die zugehörige Programmiersprache verwendet. Die bekanntesten Programmiersprachen sind Java oder Kotlin für Android Geräte, Objective-C und Swift für iOS und .Net für Windows Phone \cite{Jobe.2013}. Eine aktuelle Statistik von statcounter macht deutlich, dass die am meisten verbreiteten Betriebssysteme Android und iOS sind, weswegen im Verlauf dieser Arbeit ausschließlich auf ebendiese eingegangen wird \cite{o.V..2021b}.\\
Um Applikationen im jeweiligen App-Store veröffentlichen zu können, müssen diese eindeutig identifizierbar sein. Das erfolgt durch den Prozess des sogenannten \textit{Signings}, der ebenfalls plattformabhängig durchgeführt wird.% Ferner können Apps auch in ihrer Rohfassung auf dem Endgerät installiert werden, wenn der Entwicklermodus aktiviert ist. Das wird jedoch nicht empfohlen, da Apps aus den App Stores über eine gewisse Sicherheit verfügen, welche bei selbstinstallierten Anwendungen nicht gewährleistet werden kann.

\section{Cross-platform Applikationen}
Unter cross-platform (z.\,Dt. plattformunabhängig) Applikationen versteht man Anwendungen, die auf einer Code-Basis aufbauen und zur Laufzeit zu mehreren Anwendungen für unterschiedliche Endgeräte kompiliert werden. Außerdem können sie in herkömmlichen Quelltext-Editoren entwickelt werden, da sie im Gegensatz zu nativen Anwendungen unabhängig von bestimmten Betriebssystemen oder SDKs sind. Der Vorteil solcher Applikationen ist, dass sie, obwohl sie auf demselben Code basieren, sich komplett den plattformspezifischen Stil anpassen. Somit verringern sich auch die Entwicklungskosten, die bei Implementierung von jeweils einer Anwendung pro Betriebssystem anfallen würden. Klar abzugrenzen sind plattformunabhängige Anwendungen von hybriden Anwendungen. Das Endprodukt bei letzterem ist eine Web Applikation, die sich durch eine WebView in einem nativen Container dem Kontext (\zB Betriebssystem, Auflösung) des Geräts, in dem sie aufgerufen wird, anpasst \cite{Schickler.2015}.\\
Gängige Frameworks zur plattformunabhängigen Entwicklung sind Electron, Ionic und React Native \cite{o.V..2020}. Diese stellen meist eine begrenzte Anzahl an vorprogrammierten UI-Komponenten und Funktionalitäten zur Verfügung. Es gibt jedoch eine Vielzahl von Community Lösungen, die für wiederkehrende Problematiken eine Lösung bieten.

\section{Progressive Web Apps}
Progressive Web Apps (PWAs) sind in erster Linie Web Applikationen, also Anwendungen, die mit den üblichen Web Technologien wie HTML, CSS und ECMAScript (JavaScript) implementiert sind. Laut Mozilla Developer Network zeichnen sie sich dadurch aus, dass zusätzlich folgende technische Voraussetzungen erfüllt sind: Sie müssen auf einer sicheren Verbindung aufbauen, eine oder mehrere Service Worker besitzen und über ein App Manifest verfügen \cite{MDNcontributors.}. Ersteres wird gewährleistet durch eine HTTPS (Hypertext Transfer Protocol Secure) Verbindung, während Service Worker JavaScript-Skripte sind, die im Hintergrund auf dem Client und unabhängig von der Anwendung selbst ausgeführt werden \cite{Gaunt.}. Sie sind die wichtigste Komponente hinter den meisten Funktionalitäten, die eine PWA bietet, weswegen im Laufe dieses Kapitels genauer auf ebendiese eingegangen wird. Die letzte benötigte Komponente ist das App Manifest. Dabei handelt es sich um eine JSON-Datei, in der Einzelheiten zur Anwendung und deren Installation angegeben werden \cite{MDNcontributors.b}. Da es sich dabei, trotz der zusätzlichen nativen Funktionalitäten, um eine Web Applikation handelt, lassen sich PWAs über eine URL im Browser aufrufen und sind somit zunächst unabhängig von dem Gerät, auf dem sie aufgerufen werden. Üblicherweise werden Progressive Web Apps als Single Page Application entwickelt. Das bedeutet, dass die Anwendung aus einem einzigen HTML-Dokument besteht und Inhalte dynamisch geladen werden. Ein Vorteil davon ist, dass sich dadurch auch die Anzahl der Anfragen des Clients an den Server verringert, da nicht nach jeder Anfrage die Seite komplett neu angefordert werden muss. Der Nachteil von SPAs ist, dass im Gegensatz zu klassischen Webanwendungen Funktionalitäten wie Deep-Linking und die Navigation durch die Browser-Steuerelemente eigenständig implementiert werden müssen\cite{Wenzel.2020}. Bislang gibt es keinen offiziell definierten Web Standard für diese spezielle Art von Webanwendung.\\
Ein bedeutsames Konzept hinter Progressive Web Apps ist das „Progressive Enhancement“. Dieses verlangt, dass PWAs auf allen Endgeräten grundlegend funktionieren sollen, und schrittweise – falls der Browser und das Gerät dies unterstützt – in ihrer Funktionalität erweitert werden können \cite{Richard.}. Die Prüfung, ob der Browser die Anforderungen erfüllt, findet meist durch das Window-Objekt statt, dass das aktuelle Browser Fenster repräsentiert. Dieses enthält unter anderem das Navigator Property, das Informationen über den Browser besitzt und die Unterstützung der Service Worker API\footnote{API ist kurz für Application Programming Interface, z.Dt. Programmierschnittstelle.} signalisiert. Dadurch kann wiederum geprüft werden, ob andere Schnittstellen wie die \textit{geolocation} in dem aktuellen Browser unterstützt werden.\\
Zur Evaluierung von PWAs bietet sich \textit{Lighthouse} an, ein vorinstalliertes DevTools Add-On im Google Chrome Browser. Dort wird per Mausklick ein Testbericht zur aktuellen Anwendung erstellt, in der unter anderem auch Installierbarkeit und PWA-Optimierung geprüft werden. Dabei wird sich an die von \url{https://web.dev/pwa-checklist/} definierten Kernfunktionalitäten von PWAs orientiert, die lauten: \glqq starts fast, stays fast\grqq , \glqq works in any browser\grqq, \glqq responsive to any screen size\grqq, \glqq provides a custom offline page\grqq, und \glqq is installable\grqq. Ferner werden noch Kriterien angegeben, die die PWA optimal machen, beispielsweise \glqq can be discovered through search\grqq  oder \glqq provides context for permission requests\grqq \cite{Richard.}.

\subsection{Service Worker API}
Wie bereits erwähnt, ermöglicht der Service Worker Funktionalitäten, die Progressive Web Apps zur Konkurrenz von nativen Anwendungen machen. Dennoch kann die Service Worker API grundsätzlich in jeder Anwendung implementiert werden und ist nicht auf PWAs beschränkt, da es sich bei einem Service Worker um einen Web Worker handelt. Er fungiert dabei als eine Art Proxy Server, der zwischen der Anwendung, dem Browser und dem Netzwerk platziert ist und somit Zugriff auf Netzwerkanfragen besitzt \cite{MDNcontributors.b}. Durch seine Position hat der Service Worker jedoch keinen Zugriff auf das Document Object Model (DOM). Die Voraussetzung für die Nutzung eines Service Workers ist, dass der verwendete Browser diese unterstützt und die Verbindung über HTTPS läuft \cite{Russell.2021}. Letzteres wird damit begründet, dass Sicherheitsprobleme wie Man-in-the-middle-Angriffe, die durch die Stellung des Service Workers als Proxy begünstigt werden, vermieden werden können.\\
Ein Service Worker besitzt die Lifecycle-Events \textit{install}, \textit{activate} und unter anderem das funktionale Event \textit{fetch}. Diese ermöglichen es, auf bestimmten Ereignissen zur Laufzeit einer Anwendung zu reagieren. Der Service Worker wird üblicherweise beim ersten Aufrufen der Website registriert. Anschließend ist der Service Worker jederzeit verfügbar und läuft selbstständig im Hintergrund der Anwendung. Während des \textit{Install}-Events sollten diejenigen Dateien in den Cachespeicher aufgenommen werden, die sich im Laufe der Anwendung nicht ändern. Das betrifft beispielsweise Styling Sheets, das App Manifest und Bild Dateien \cite{MDNcontributors.d}. Außerdem kann auch die index.html, die als Eintrittspunkt von Single-Page-Webanwendungen dient, im Cache abgelegt werden. Das \textit{Active}-Event wird dafür genutzt, veraltete Cachespeicherinhalte zu bereinigen und vorherige Service Worker Registrierungen zu entfernen. Um auf Netzwerkanfragen zu reagieren, gibt es das \textit{Fetch}-Event. Hier ist es möglich, angeforderte Ressourcen ebenfalls im Cache abzulegen. Dafür gibt es verschiedene Strategien, zwischen denen je nach Anwendungsfall der Applikation abgewägt werden kann. Gängige Strategien sind der \textit{Cache First}- oder der \textit{Cache then network}-Ansatz \cite{Liebel.2019}. Durch weitere funktionale Events wie \textit{push}, \textit{notificationclick} und \textit{sync} und die Nutzung von alten und neuen Programmierschnittstellen ermöglicht der Service Worker das moderne, native-like Web. Jede dieser APIs sollten ebenfalls im Sinne von \textit{progressive enhancement} eingebunden werden. Im Laufe dieser Arbeit wird genauer auf die Cache API, die Notification API, die Push API sowie die Geolocation API eingegangen.\\
Für alle der genannten Programmierschnittstellen ist es nötig, die Erlaubnis des Nutzers zu erfragen. Zugriff und Verwaltung aller erteilten Erlaubnisse bietet die Permissions API. Diese verfügt über ein Permission Registry, das Permissions für Schnittstellen wie \textit{geolocation}, \textit{bluetooth}, \textit{speaker} und \textit{device-info} enthält. Laut dessen W3 Spezifikation gibt es drei Status der Erlaubnis: \textit{granted}, \textit{denied} und \textit{prompt}. Außerdem wird aufgrund des hohen Einflusses von Permissions unterschieden zwischen Funktionalitäten, die in unsicheren Kontexten (HTTPS) und jenen, die nur im sicheren Kontext verwendet werden können \cite{Lamouri.2020}.
Mittlerweile bietet Google, unter anderem zur vereinfachten Implementierung und Verwaltung von Service Workern, das Tool \textit{Workbox} an. Es handelt sich dabei um eine Bibliothek, die die gängigsten Funktionalitäten von Service Workern zur Verfügung stellt, wodurch wiederkehrende Prozesse eliminiert werden.

\section{React}
React ist eine von Facebook entwickelte, open-source JavaScript-Bibliothek, die seit 2013 publiziert ist. Sie zeichnet sich dadurch aus, dass sie in erster Linie zum Erstellen von User Interfaces entwickelt wurde. Durch ReactDOM-Bibliothek, wird die Anwendung um das Rendern dieser Benutzeroberflächen der erweitert. Daran lässt sich auch erklären, warum React im Gegensatz zu Vue oder Angular eigentlich kein Framework ist. Legt man Projekte mit diesen Frameworks an, erhält man eine Vielzahl von eingebauten Werkzeugen zum entwickeln von skalierbaren, mächtigen Webanwendungen. Im Gegensatz dazu ist React als UI-Bibliothek leichtgewichtig und ermöglicht individuelle Erweiterung zur Anpassung an die Anforderungen des Projekts \cite{Barger.2021}.\\
Die einfachste Möglichkeit React in einem Projekt zu nutzen ist, es über eine CDN\footnote{Content Delivery Network} einzubinden. Dabei ist es ein Anliegen der Entwickler, dass nur so wenig React genutzt werden kann, wie benötigt. Ferner ist es möglich React über Package-Manager wie npm in ein Projekt zu importieren oder durch Toolchains\footnote{Eine Sammlung von Werkzeugen, die zum unkomplizierten Aufsetzen eines Produkts dienen.} wie Create-React-App über die CLI\footnote{Command Line Interface, z. Dt. Kommandozeile} eine Single Page Application zu erstellen \cite{Facebook.g}. Im Codeausschnitt \ref{lst:basicReact} ist ein Beispiel zu sehen, wie eine React Anwendung in ihrer kleinsten Form aussehen kann. Zuerst müssen die Module React und ReactDom importiert werden. Dann wird per Aufruf der \textit{React.createElement(...)}-Funktion mit den Übergabeparametern HTML-Element, Attribute und Inhalt ein React Element erstellt. Zuletzt wird in Zeile 9 die \textit{ReactDOM.render(...)}-Funktion genutzt, um das erstellte Element einem anderen Element zuzuordnen und damit eine Hierarchie zu erzeugen. In diesem Fall konkret dem HTML-Element mit der \textit{id} root. Die \textit{render}-Funktionen kann als weiteren Übergabeparameter die \textit{Properties} eines Elements oder einer Komponente enthalten, worauf im Laufe des Kapitels genauer eingegangen wird.

\begin{lstlisting}[language=Java,caption={Schlichtes Beispiel der index.js einer React Applikation},captionpos=b,label={lst:basicReact}]
import React from 'react';
import ReactDOM from 'reactdom';

var element = React.createElement(
	'h1',
	{ className: 'greeting},
	'Hello world.'
);
ReactDOM.render(element, document.getElementById('root');
\end{lstlisting}

Eines der Argumente zur Nutzung eines Programmiergerüsts wie React ist dessen Implementierung des Virtual Document Object Model. Dieses baut auf dem normalen DOM auf und ermöglicht es, dass nur diejenigen UI-Elemente neu gerendert werden, deren Daten sich verändert haben. Des weiteren bietet sich durch die Abkapselung in Komponenten ein hohes Maß an Wiederverwendbarkeit. Außerdem ist React wie bereits erwähnt leichtgewichtig, da es sich bei der Hauptbibliothek nur um die Implementierung der wichtigsten Bestandteile handelt. Weitere Funktionalitäten wie der React Router zur Programmierung von der Navigation in einer SPA oder anderen Bibliotheken können nach Bedarf importiert werden. Ebenso gibt es UI-Bibliotheken wie MaterialUI oder PrimeReact für React, die häufig implementierte Komponenten im modernen Design anbieten. Generell lassen sich Progressive Web Apps jedoch mit jedem Framework oder auch mit einer einfachen Vanilla-JavaScript Implementierung verwirklichen.\\
Kritik erlangt React vor allem wegen des Vorwurfs, dass gegen das Entwurfsprinzip der Trennung der Verantwortlichkeiten (Separation of Concerns) verstößt. Dies wird in der Webentwicklung so umgesetzt, dass verschiedene Technologien wie HTML, CSS und JavaScript jeweils in eigenen Dateien modelliert oder programmiert werden. Im Gegensatz dazu steht jedoch Reacts Syntax Erweiterung JSX. Diese ermöglicht die drei genannten Technologien innerhalb von JavaScript zu entwickeln. Ein Beispiel dafür ist in \ref{lst:label} zu sehen. Wichtig ist hier jedoch, dass die Zeile 8 auch in JavaScript geschrieben werden kann, da es sich hierbei letztendlich um den Aufruf der React.createElement(component, props, …children)-Funktion handelt \cite{Facebook.h}.\\

\begin{lstlisting}[language=Java,caption={Nutzung von JSX},captionpos=b,label={lst:label}]
import React from 'react';
import ReactDOM from 'reactdom';

const Example = (props) =>  {
	const greeting = 'Hello world'
	
	return (
		<h1>{{ greeting }}</h1>
	)	
}
\end{lstlisting}

Außerdem gibt es einige Änderungen, die sich durch diese Art zu programmieren ergeben. Beispielsweise kann in JSX auf das Semikolon am Ende einer Zeile verzichtet werden und die CSS-Klasse \textit{class} nennt sich \textit{className}. Letzteres ist eines von mehreren Syntaxänderungen bei JSX. Das liegt dem zugrunde, dass jeder JSX-Code in JavaScript-Code umkompiliert wird. Das Schlüsselwort \textit{class} ist dabei in JavaScript ein reserviertes Wort für Klassen und nicht für eine CSS-Klasse. Ein weiteres Beispiel ist \textit{htmlFor} statt \textit{for}. Durch diese Syntax bietet React eine inklusive Dateistruktur. Die Entwickler begründen das damit, dass es hier im Gegensatz zu anderen JS-Frameworks, in denen es pro Komponente jeweils eine getrennte HTML-, CSS- und JavaScript-Datei gibt, lediglich um eine Trennung der Technologien, nicht aber der Verantwortlichkeiten, handelt. Diese Syntax hingegen verbinden die Render Logik enger mit den Benutzeroberfläche und führt bei einer korrekten Aufteilung in Komponenten zu einer starken Kohäsion. Das bietet dem Entwicklern mehr Übersichtlichkeit und Verständnis für Zusammengehörigkeit. Auf der anderen Seite werden komplexe Komponenten jedoch durch diese inklusive Struktur schnell unübersichtlich, weshalb es sinnvoll ist, eine Aufteilung in Unterkomponenten angelehnt an deren Funktionalität vorzunehmen. Wichtig ist hierbei auch eine organisierte Ordnerstruktur aufrecht zu erhalten, damit die Anwendung wartbar bleibt.\\
Im Folgenden wird genauer auf einige Grundkonzepte von React eingegangen, wobei React-spezifische Begriffe bewusst nicht übersetzt werden.

\subsection{Komponenten}
Wenn Teile des Codes abgekapselt und wiederverwendet werden sollen, wird eine Komponente erstellt, die meist in einer Datei mit demselben Namen implementiert wird. React unterscheidet dabei zwischen zustandslosen und klassenbasierten Komponenten. Ersteres bezeichnete ehemals Komponenten, die nur zur Darstellung von zusammengehörigen UI-Elementen genutzt wird. Sie sind schlank, wiederverwendbar und leicht zu warten. Klassenbasierte Komponenten hingegen basieren auf herkömmlichen E6-Klassen und bieten sogenannte States, die lokale Daten einer Komponente verwalten, und einen Lifecycle. Die Hooks API bietet jedoch seit React 16.8 ein neues Konzept an, um States in sogenannten funktionalen Komponenten zu organisieren und somit die Vorteile von zustandslosen und klassenbasieren Komponenten zu vereinen. Im nächsten Kapitel wird genauer auf die Funktionsweise von Hooks eingegangen.\\
Zur Kommunikation und Datenaustausch zwischen Eltern- und Kind-Komponenten werden Properties und Callback-Methoden genutzt. Ein Property ist eine Art Übergabeparameter, die von der Eltern- an die Kindkomponente weitergegeben wird. Das Kind kann mit diesen Daten die eigene UI und Funktionalität entwickeln oder wiederum der eigene Kindkomponente geben. Wichtig ist dabei, dass übergebene Informationen lediglich gelesen, nicht aber verändert werden sollen. Die Kommunikation nach außen funktioniert über Events. In der Kindkomponente wird dafür eine Callback-Methode aufgerufen und somit signalisiert, dass die Eltern denjenigen Code ausführen sollen, der als Reaktion auf die Veränderung in der Kindkomponente dient.

\subsection{Hooks}
In Version 16.8 erfolgte die Einführung von Hooks. Diesen bieten die Möglichkeit, States und andere React Funktionalitäten zu implementieren, ohne eine JavaScript Klasse deklarieren zu müssen. Dadurch vereinfachen sich Komponenten, die ehemals von der \textit{Component}-Klasse abgeleitet wurden, um in Konstruktoren Zustände definieren zu können. Der Rückgabewert dieser funktionalen Komponenten ist die UI-Deklaration selbst \cite{Facebook.d}.\\
Es gibt unterschiedliche Arten von Hooks, die gängigsten sind der \textit{useState}- und der \textit{useEffect}-Hook. Sie werden erstmalig direkt nach dem Rendern der Komponente ausgeführt. Außerdem gibt es die Möglichkeit, eigene Hooks zu definieren.\\
Die useState-Hook dient zur Deklaration eines lokalen States. In Zeile 4 des folgenden Beispiels \ref{lst:useState} wird der State \textit{counter} in der Komponente Example initialisiert. Dieser erhält den Standardwert 0 und verfügt über einen Getter -- hier genannt \textit{counter} -- und den Setter, genannt \textit{setCounter()}, über diese Funktion der Zustand geändert werden kann.\\
\begin{lstlisting}[language=Java,caption={Beispiel der Nutzung von useState},captionpos=b,label={lst:useState}]
import React, { useState } from "react";

const Example = (props) =>  {
	const [counter, setCounter] = useState(0);
}
\end{lstlisting}

Der useEffect-Hook ersetzt die Lifecycle-Methoden \textit{componentDidUpdate}, \textit{componentDidMount} und  \textit{componentWillUnmount} der klassenbasierten Komponenten. Es bietet sich deshalb an, Ressourcenanfragen hier zu behandeln. Das Beispiel in \ref{lst:useEffect} zeigt einen einfachen Effect welcher die frühere \textit{componentDidMount}-Funktion und \textit{componentWillUnmount()}-Funktion ersetzt. Der erste Übergabeparameter des useEffect-Aufrufs ist eine Funktion, die ausgeführt werden soll und der Zweite ein Array, genannt \textit{dependency array}. Wenn das Array leer ist, bedeutet das, dass der Effect nur einmal nach dem Rendern der Komponente ausgeführt werden soll. Durch jedes Element, das diesem Array hinzugefügt wird, startet erneut diejenige Funktion, die als erster Übergabeparameter übergeben wurde.\\

\begin{lstlisting}[language=Java,caption={Beispiel für Nutzung der useEffect-Hook},captionpos=b,label={lst:useEffect}]
import React, { useEffect } from "react";

const Example = (props) =>  {
	const [counter, setCounter] = useState(0);

	useEffect(() => {
		console.log('Component did render!);
	}, []);

	useEffect(() => {
		console.log('Counter was updated!);
	}, [counter]);
	
	// setCounter() is called somewhere in the component
}
\end{lstlisting}

Durch das \textit{dependency array} haben Effects den Vorteil, dass sie sehr fallspezifisch auf Änderungen der Daten reagieren können und somit umso mehr den dynamischen Gedanken der Single Page Applikation realisieren.

\section{React Native}
React Native ist eines der meist genutzten Frameworks zur Entwicklung von Native Apps \cite{o.V..2021b}. Es basiert auf React und ist ebenfalls von Facebook entwickelt und veröffentlicht worden. Bekannte Apps, die auf React Native basieren, sind selbstverständlich die Facebook und Instagram App, aber auch die Unterkunftbuchungsapp Airbnb oder die Lieferdienstapp UberEats \cite{Facebook.f}.\\
Wie bereits im Kapitel Grundlagen erklärt, werden native Applikationen üblicherweise mit der dafür vorgesehenen Programmiersprache implementiert. React Native nutzt die Möglichkeit, mit JavaScript Schnittstellen und Views von Native Anwendungen anzusprechen und das Ganze somit als native Code zu rendern. Zur Laufzeit werden dabei aus den \textit{Core Components} von React Native die jeweils korrespondierenden \textit{Native Components} von Android, iOS und anderen Betriebssystemen erstellt\cite{Facebook.}. Das wird ermöglicht durch die Architektur von React Native, die in drei Komponenten unterteilt ist: eine Laufzeitumgebung für JavaScript auf der Zielplattform, eine Bridge und ein Native Module. Ersteres bietet in iOS JavaScriptCore und in Android eine Lösung, die von React Native zur Verfügung gestellt wird. Dabei gibt es in React Native plattformübergreifende Abstraktionen wie \textit{View}, \textit{Image} oder \textit{NetInfo}, die korrespondierende Native Komponenten besitzen. Die Bridge ist für die Kommunikation zwischen Native Module und JS-Umgebung verantwortlich. Sie erhält von ersterem Events wie Toucheingaben oder Netzwerkanfragen und vom Native Module Anweisungen für das UI oder APIs. Der Austausch findet dabei asynchron über JSON\footnote{JavaScript Object Notation, ein programmiersprachen-unabhängiges Datenformat} statt. Der letzte Teil der Architektur ist das Native Module. Dies ist die Anwendung, die der Nutzer verwendet \cite{Behrends.2018}. Es handelt sich somit bei Apps, welche mit React Native programmiert wurden, nicht um hybride Apps, da sie nicht in einem Native Container gerendert, sondern tatsächlich in nativen Code umgewandelt werden. Die Entwicklung solcher Apps ist dennoch plattformunabhängig, da die Implementierung in JavaScript erfolgt und die Umwandlung erst zur Laufzeit geschieht. React Native Applikationen lassen sich dadurch schwierig in die bereits erklärten Kategorien von mobilen Anwendungen einordnen.\\
Zur Nutzung von React Native Apps benötigt das Gerät mindestens iOS 11.0 oder Android 5.0 \cite{Facebook.d}. Genau wie andere Native Apps können sie in den plattform-spezifischen App Store veröffentlicht und dort vom Nutzer heruntergeladen, installiert und aktualisiert werden. Dabei ist der Signing-Prozess derselbe wie bei Native Applikationen.
Zur Implementierung von Komponenten und Funktionalitäten in React Native gibt es mehrere Optionen. Die erste Option ist das Nutzen der bereits erwähnten Core Components. Weiterhin kann auf Community Lösungen zugegriffen werden. Sollten diese nicht ausreichen, ist dies meist die effizienteste Wahl, da viele Funktionalitäten bereits zufriedenstellend und stabil entwickelt und veröffentlicht wurden. Unter \url{https://reactnative.directory/} sind diese auffindbar und durch einen errechneten Directory Score nach mehreren Kriterien wie Beliebtheitskurve, Sterne auf GitHub oder Anzahl der Downloads bewertet. Die letzte Option ist es, Native Modules selbstständig zu programmieren.




%Um einen schnellen Einstieg in die Entwicklung von Apps zu ermöglichen, bietet React Native die sogenannte Expo CLI. Diese läuft mit der zugehörigen Expo Go iOS oder Android App, indem die entwickelte App in Echtzeit über einen Packager gerendert wird. Der Nachteil dieses Ansatzes ist, dass mit Expo kein nativer Code für die genannten Betriebssysteme programmiert werden kann und somit die Möglichkeiten auf die von Expo gebotenen APIs beschränkt sind. Falls diese nicht ausreichen, kann man ein mit Expo aufgesetztes JS Projekt durch den Befehl „eject expo“ einfach in ein natives iOS und Android Projekt umwandeln. Dafür werden nötige Abhängigkeiten heruntergeladen und der „android“ und „ios“ Ordner erstellt, indem nativer Code programmiert wird. Das Ergebnis wird wiederum durch ExpoKit verwaltet und bezeichnet sich als „bare Workflow“ %() \cite{Facebook.c}. Es handelt sich dabei um eine pure React Native App, die immer noch alle genutzten Expo SKD APIs beinhaltet. Auch die Expo Go App kann weiterhin zum Rendern verwendet werden. (https://docs.expo.io/bare/exploring-bare-workflow/#adding-a-library-from-the-expo-sdk).

%-	Warum RN zur Entwicklung von CPA?
%o	RN sollte genutzt werden, wenn möglichst viele Native Controls genutzt werden
%-	Wie wird aus RN JS Anwendung eine native App? Detaillierte Erklärung!! Einfach in WebView oder wird tatsächlich alles in nativen Code umgewandelt?
%o	Zur laufzeit macht RN aus seinen React Komponents die korrespondieren Android / iOS Elemente (https://reactnative.dev/docs/intro-react-native-components)
%o	RN hat eine Liste von Core Komponenten, es können auch eigene Native Komponenten gebaut werden oder Komponenten aus der Community genutzt werden
%	Zb View von RN wird zu ViewGroup in Android und UIView in iOS
%o	Komponenten können aus Sammlung von Core Komponents / eigenen gemacht werden (https://reactnative.dev/docs/intro-react)
%-	Über Plattform kann abhängig vom System Sachen gerendert werden (https://reactnative.dev/docs/platform-specific-code)
%-	Strong Community und open source
%-	Abhängigkeiten der CLI!
%-	Zugriff nur auf bestimmte beschränkte Funktionalitäten
